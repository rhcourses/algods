\begin{frame}
    \begin{block}{Sortieren durch Einfügen}
        \begin{itemize}
            \item Ansatz: Nimm das nächstbeste Element und füge es an der passenden Stelle ein.
        \end{itemize}
    \end{block}
    \begin{block}<2->{Vorteil}
        \begin{itemize}
            \item schnell für kurze Listen
            \item einfach zu verstehen und zu implementieren.
            \item typischer Aufräum-Ansatz
        \end{itemize}
    \end{block}
    \begin{block}<3->{Komplexität}
        \begin{itemize}
            \item Quadratisch in der Länge der Liste  (Schreibe: \alert{\osquare}).
            \item Bei Länge $n$ müssen $n$ Elemente einsortiert werden.
            \item Jedes Einsortieren dauert bis zu $n$ Schritte.
        \end{itemize}
    \end{block}
\end{frame}

\begin{frame}
    \begin{block}{Sortieren durch Einfügen}
        \begin{itemize}
            \item Ansatz: Nimm das nächstbeste Element und füge es an der passenden Stelle ein.
        \end{itemize}
    \end{block}
    \begin{block}<2->{typische Implementierung für das Einsortieren eines Elements}
        \begin{itemize}
            \item Füge das nächste Element am Ende der Liste an.
            \item Tausche es solange nach links, bis es größer als sein linker Nachbar ist.
        \end{itemize}
    \end{block}
    \begin{block}<3->{Beobachtung}
        \begin{itemize}
            \item kann sehr effizient \alert{in place} umgesetzt werden.
            \item d.h. ohne eine separate Hilfsliste
        \end{itemize}
    \end{block}
\end{frame}