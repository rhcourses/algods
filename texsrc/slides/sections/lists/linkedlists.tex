\begin{frame}
    \begin{block}{Attribute eines Listenelements}
        \begin{itemize}
            \item Datensatz
            \item Zeiger/Referenzen auf die Nachbarelemente
        \end{itemize}
    \end{block}
    \begin{block}<2->{Zwei typische Varianten}
        \begin{itemize}
            \item einfach verkettete Liste
            \item doppelt verkettete Liste
        \end{itemize}
    \end{block}
    \begin{block}<3->{Anhängen von Elementen}
        \begin{itemize}
            \item Ende der Liste suchen
            \item neues Element anhängen
        \end{itemize}
    \end{block}
\end{frame}

\begin{frame}
    \begin{block}{Anhängen von Elementen}
        \begin{itemize}
            \item Ende der Liste suchen
            \item neues Element anhängen
        \end{itemize}
    \end{block}
    \begin{block}<2->{Markierung des Listen-Endes: \alert{Sentinel-Prinzip}}
        \begin{itemize}
            \item Verwendung eines \emph{Dummy-Elements}
            \item wird nicht für Daten verwendet
            \item markiert das Ende der Liste
        \end{itemize}
    \end{block}
    \begin{block}<3->{Vorteil des Dummy-Elements}
        \begin{itemize}
            \item keine Sonderbehandlung der leeren Liste notwendig
        \end{itemize}
    \end{block}
\end{frame}

\begin{frame}
    \begin{block}{Implementierung der ganzen Liste}
        \begin{itemize}
            \item Ein Listenelement ist gleichzeitig auch eine Liste.
            \item \alert{Listen haben eine rekursive Struktur!}
            \item Container-Klasse für Liste ist dennoch oft nützlich
        \end{itemize}
    \end{block}
    \begin{block}<2->{Attribute einer verketteten Liste}
        \begin{itemize}
            \item Zeiger/Referenz auf Anfang der Liste oder Dummy
        \end{itemize}
    \end{block}
    \begin{block}<3->{Vorteil der Container-Klasse}
        \begin{itemize}
            \item kann Dummy vor Benutzer verstecken
            \item manche Operationen einfacher umsetzbar
        \end{itemize}
    \end{block}
\end{frame}
