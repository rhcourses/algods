\begin{frame}
    \begin{block}{Attribute eines dynamischen Arrays}
        \begin{itemize}
            \item Array für die Daten
            \item tatsächliche und maximale Länge
        \end{itemize}
    \end{block}
    \begin{block}<2->{Zentrale Operation: \alert{\code{reallocate}}}
        \begin{itemize}
            \item neues Array für die Daten mit neuer Länge erzeugen
            \item alle Elemente an die neue Stelle kopieren
            \item maximale Länge aktualisieren
        \end{itemize}
    \end{block}
    \begin{block}<3->{Anhängen von Elementen}
        \begin{itemize}
            \item Element an erste freie Stelle schreiben
                  und Größe aktualisieren
            \item ggf. vorher \code{reallocate} durchführen
        \end{itemize}
    \end{block}
\end{frame}

\begin{frame}
    \begin{block}{Anhängen von Elementen}
        \begin{itemize}
            \item falls Array voll: \code{reallocate} durchführen
            \item neues Element an erste freie Stelle schreiben
            \item Größe aktualisieren
        \end{itemize}
    \end{block}
    \begin{block}<2->{Wie viel Speicher sollte bei \code{reallocate} reserviert werden?}
        \begin{itemize}
            \item<3-> Antwort: Z.B. immer verdoppeln.
            \item<3-> Ziel: Der Speicher muss exponentiell wachsen,
            damit \code{reallocate} nicht zu oft notwendig ist.
        \end{itemize}
    \end{block}
\end{frame}
