\begin{frame}
    \begin{defblock}{Binärbaum}
        Für jeden Binärbaum gilt eine der folgenden Möglichkeiten:
        \begin{itemize}
            \item Der Baum ist leer.
            \item Der Baum besteht aus einer Wurzel und zwei Teil\alert{bäume}n.
        \end{itemize}
    \end{defblock}
    \begin{block}<2->{Bemerkungen}
        \begin{itemize}
            \item Rekursive Definition beschreibt direkt die Struktur.
            \item Verallgemeinerung zu \emph{Bäumen} möglich.
            \item Häufig vorkommende Struktur in Mathematik und Informatik.
        \end{itemize}
    \end{block}
\end{frame}

\begin{frame}
    \begin{block}{Beispiele}
        \begin{itemize}
            \item Turnierbäume bei Wettbewerben
            \item Wahrscheinlichkeitsbäume
            \item Stammbäume
            \item Modellierung von Abhängigkeiten
        \end{itemize}
    \end{block}
    \begin{block}<2->{Anwendung in der Informatik}
        \begin{itemize}
            \item Strukturierung und Sortierung von Daten
        \end{itemize}
    \end{block}
\end{frame}
