\begin{frame}
    \begin{block}{Ziel: Optimierung von binären Suchbäumen}
        \begin{itemize}
            \item Problem: Bäume können aus der Balance geraten.
            \item Lösung: Reorganisieren, wenn das Ungleichgewicht zu groß wird.
        \end{itemize}
    \end{block}
    \begin{block}<2->{Vorgehensweise}
        \begin{itemize}
            \item Beim Einfügen oder Löschen Struktur des Baumes analysieren.
            \item Bei Bedarf Knoten umsortieren, damit linker und rechter Teilbaum
                  jedes Knotens ungefähr gleich groß sind.
            \item Problem: Analyse darf nicht zu teuer sein.
            \item Frage: Was bedeutet \gqq{ungefähr gleich groß}überhaupt?
        \end{itemize}
    \end{block}
\end{frame}

\begin{frame}
    \begin{defblock}{Tiefe}
        Die Tiefe eines Knotens ist die Länge des Pfades von der Wurzel bis zu diesem Knoten.
    \end{defblock}
    \begin{block}<2->{Bemerkungen}
        \begin{itemize}
            \item Berechnung rekursiv:
            \begin{itemize}
                \item Tiefe der Wurzel: $0$.
                \item Tiefe eines Knotens: 1 + Tiefe des Elternknotens
            \end{itemize}
            \item Kann beim Einfügen/Löschen nebenbei berechnet werden.
            \item Kann bei Bedarf im Knoten gespeichert und gepflegt werden.
        \end{itemize}
    \end{block}
\end{frame}

\begin{frame}
    \begin{defblock}{Höhe}
        Die Höhe eines Baums ist die Tiefe des tiefsten Knotens im Baum.
    \end{defblock}
    \begin{block}<2->{Bemerkungen}
        \begin{itemize}
            \item Berechnung rekursiv:
            \begin{itemize}
                \item Höhe eines Blatts Wurzel: $1$.
                \item Höhe eines Knotens: 1 + Höhe des höheren Kindes
            \end{itemize}
            \item Kann beim Einfügen/Löschen nebenbei berechnet werden.
            \item Kann bei Bedarf im Knoten gespeichert und gepflegt werden.
        \end{itemize}
    \end{block}
\end{frame}

\begin{frame}
    \begin{defblock}{Balancefaktor}
        Der Balancefaktor eines Knotens ist die Differenz zwischen der Höhe des rechten und des linken Teilbaums.
    \end{defblock}
    \begin{block}<2->{Bemerkungen}
        \begin{itemize}
            \item Kann beim Einfügen/Löschen nebenbei berechnet werden.
            \item Gutes Maß für die Ausgeglichenheit des Baumes.
        \end{itemize}
    \end{block}
\end{frame}

\begin{frame}
    \begin{defblock}{AVL-Baum}
        Ein AVL-Baum ist ein binärer Suchbaum mit folgender Eigenschaft:
        \begin{itemize}
            \item Der Balancefaktor jedes Knotens liegt im Intervall $\lbrack -1, 1 \rbrack$.
        \end{itemize}
    \end{defblock}
    \begin{block}<2->{Umsetzung}
        \begin{itemize}
            \item Beim Einfügen/Löschen Balancefaktoren bestimmen.
            \item Nach Einfügen in Teilbaum: Umorganisieren der Knoten.
            \item<3-> Nach Einfügen in Teilbaum bedeutet: Nach Rekursion.
        \end{itemize}
    \end{block}
\end{frame}

\begin{frame}
    \begin{block}{Analyse der Balancefaktoren}
        Nach Einfügen eines Knotens Balancefaktor prüfen.
        \begin{itemize}
            \item Falls $-2$: Linker Teilbaum zu hoch.
            \item Falls $2$: Rechter Teilbaum zu hoch.
            \item Prüfe Balancefaktor des linken/rechten Teilbaumes.
            \item Führe passende \emph{Rotation} durch.
        \end{itemize}
    \end{block}
    \begin{block}<2->{Ungleichgewichts-Situationen und Rotationen}
        \begin{itemize}
            \item Links-Links
            \item Links-Rechts
            \item Rechts-Rechts
            \item Rechts-Links
        \end{itemize}
    \end{block}
\end{frame}

\begin{frame}    
    \begin{block}{Verhalten beim Einfügen/Löschen/Suchen}
        \begin{itemize}
            \item Linker und rechter Teilbaum in allen Knoten fast gleich tief.
            \item Logarithmisches Verhalten beim Suchen, Einfügen und Löschen.
        \end{itemize}
    \end{block}
\end{frame}
