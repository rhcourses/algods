\begin{frame}
    \begin{block}{Ziel: Suche nach dem größten Produkt benachbarter Elemente einer Liste}
        \begin{itemize}
            \item Gegeben: Eine Liste von Zahlen der Länge $n$.
            \item Ergebnis: Das größte Produkt von $m$ benachbarten Elementen.
        \end{itemize}
    \end{block}
    \begin{block}<2->{Naiver Ansatz}
        \begin{itemize}
            \item Durchlaufe die Liste von Stelle $0$ bis $n-m$.
            \begin{itemize}
                \item Von jeder Position aus berechne das Produkt der nächsten $m$ Elemente.
            \end{itemize}
        \end{itemize}
    \end{block}
    \begin{block}<3->{Komplexität}
        \begin{itemize}
            \item $n-m \leq n$ Durchläufe der äußeren Schleife
            \item pro Durchlauf: $m$ Durchläufe der inneren Schleife
            \item Komplexitätsklasse: \bigo{n \cdot m} (für große $m$ grob \osquare)
        \end{itemize}
    \end{block}
\end{frame}

\begin{frame}
    \begin{block}{Ziel: Suche nach dem größten Produkt benachbarter Elemente einer Liste}
        \begin{itemize}
            \item Gegeben: Eine Liste von Zahlen der Länge $n$.
            \item Ergebnis: Das größte Produkt von $m$ benachbarten Elementen.
        \end{itemize}
    \end{block}
    \begin{block}<2->{Optimierter Ansatz}
        \begin{itemize}
            \item Berechne das Produkt der ersten $m$ Elemente.
            \item Durchlaufe die Liste von Stelle $m$ bis $n$.
            \begin{itemize}
                \item Multipliziere das Produkt mit dem Element an Stelle $i$.
                \item Dividiere das Produkt durch das Element an Stelle $i-m$.
            \end{itemize}
        \end{itemize}
    \end{block}
    \begin{block}<3->{Komplexität}
        \begin{itemize}
            \item Berechnung des Anfangsprodukts: \olin[m]
            \item Schleife \olin[n-m]
            \item \alert{Gesamt: \olin}
        \end{itemize}
    \end{block}
\end{frame}
