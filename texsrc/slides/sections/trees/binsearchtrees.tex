\begin{frame}
    \begin{defblock}{Binärer Suchbaum}
        Ein binärer Suchbaum ist ein Binärbaum mit folgenden Eigenschaften:
        \begin{itemize}
            \item Die Knoten enthalten \emph{Schlüssel} und \emph{Werte} (engl. \emph{key/value}).
            \item Auf den Schlüsseln ist eine \emph{totale Ordnung} definiert.
            \item<2-> Für jeden Knoten gilt:
            \begin{itemize}
                \item Die Schlüssel aller Knoten im linken Teilbaum sind kleiner als der Schlüssel der Wurzel.
                \item Die Schlüssel aller Knoten im rechten Teilbaum sind größer als der Schlüssel der Wurzel.
                \item (Gleichheit muss ggf. einer der Seiten zugeschlagen werden.)
            \end{itemize}
        \end{itemize}
    \end{defblock}
    \begin{block}<3->{Ziel: Effiziente Implementierung von Listen und Datenbanken}
        \begin{itemize}
            \item Idee: Binärer Suche und effiziente Sortierverfahren
                        direkt in einer Datentstruktur verankern.
        \end{itemize}
    \end{block}
\end{frame}

\begin{frame}
    \begin{block}{Auffinden von Elementen mit einem bestimmten Suchschlüssel}
        \begin{itemize}
            \item<2-> Baum leer: Nicht gefunden.
            \item Suchschlüssel in Wurzel gefunden, liefere (Wert der) Wurzel.
            \item Suchschlüssel kleiner als Wurzel: Suche im linken Teilbaum.
            \item Suchschlüssel größer als Wurzel: Suche im rechten Teilbaum.
        \end{itemize}
    \end{block}
    \begin{block}<3->{Einfügen von Elementen mit einem bestimmten Suchschlüssel}
        \begin{itemize}
            \item<4-> Baum leer: Hier einfügen.
            \item Suchschlüssel kleiner Wurzel: Füge in linken Teilbaum ein.
            \item Suchschlüssel größer Wurzel: Füge in rechten Teilbaum ein.
        \end{itemize}
    \end{block}
\end{frame}

\begin{frame}
    \begin{block}{Löschen von Elementen mit einem bestimmten Suchschlüssel}
        \begin{itemize}
            \item Suche das zu löschende Element.
            \item Falls Blatt: Entfernen.
            \item Ansonsten: Suche den direkten Nachfolger oder Vorgänger.
            \item Vertausche Element mit Nachfolger/Vorgänger.
            \item Entferne Element aus entsprechendem Teilbaum.
        \end{itemize}
    \end{block}
\end{frame}

\begin{frame}    
    \begin{block}{Verhalten im Optimalfall}
        \begin{itemize}
            \item Linker und rechter Teilbaum in allen Knoten gleich tief.
            \item Logarithmisches Verhalten beim Suchen, Einfügen und Löschen.
        \end{itemize}
    \end{block}
    \begin{block}<2->{Verhalten im Worst Case}
        \begin{itemize}
            \item Eine Seite hat starkes Übergewicht.
            \item Extremfall: Jeder Knoten hat nur einen Teilbaum.
            \item Der Baum ist dann de facto eine Liste.
        \end{itemize}
    \end{block}
\end{frame}
