\begin{frame}
    \begin{block}{Bisher: Informelle Komplexitätsabschätzungen}
        \begin{itemize}
            \item Laufzeitabschätzungen in Abhängigkeit der Größe einer Datenstruktur
            \begin{itemize}
                \item z.B. Länge einer Liste oder Anzahl der Elemente eines Baumes
            \end{itemize}
            \item<2-> Beobachtung: Laufzeit wird i.d.R. \emph{ungenau} angegeben.
            \begin{itemize}
                \item z.B. Schleifendurchläufe zählen, aber nicht die Anzahl der Operationen innerhalb der Schleife
                \item z.B. geschachtelte Schleifen berücksichtigen, hintereinander ausgeführte Schleifen aber nicht
            \end{itemize}
        \end{itemize}
    \end{block}

    \begin{block}<3->{Ziel: Formalisierung dieser Ungenauigkeiten}
        \begin{itemize}
            \item Wie kommen diese Abschätzungen zustande?
            \item Welche Operationen müssen gezählt werden?
        \end{itemize}
    \end{block}
\end{frame}

\begin{frame}
    \begin{exblock}{Maximum einer Liste bestimmen}
        \inputsrcfile[slidelisting, basicstyle=\ttfamily\scriptsize]{sections/complexity/bigo/code/searchmax.go}[4-12]
    \end{exblock}
    \begin{block}<2->{Komplexität}
        \begin{itemize}
            \item $n$ Schleifendurchläufe (Vergleiche \code{v > max})
            \item Komplexitätsklasse: \olin
        \end{itemize}
    \end{block}
\end{frame}

\begin{frame}
    \begin{exblock}{Differenz zw. Minimum und Maximum bestimmen}
        \inputsrcfile[slidelisting, basicstyle=\ttfamily\scriptsize]{sections/complexity/bigo/code/diffminmax.go}[4-6]
    \end{exblock}
    \begin{block}<2->{Komplexität}
        \begin{itemize}
            \item Je ein Durchlauf von \code{searchMax} und \code{searchMin}
            \begin{itemize}
                \item Diese haben jeweils lineare Komplexität (\olin).
            \end{itemize}
            \item Komplexitätsklasse: \olin
            \begin{itemize}
                \item \alert{Warum nicht \bigo{2n}?}
            \end{itemize}
        \end{itemize}
    \end{block}
\end{frame}

\begin{frame}
    \begin{exblock}{Minimale Differenz von Elementen bestimmen}
        \inputsrcfile[slidelisting, basicstyle=\ttfamily\scriptsize]{sections/complexity/bigo/code/closestpair.go}[6-17]
    \end{exblock}
    \begin{block}<2->{Komplexität}
        \begin{itemize}
            \item $n$ Durchläufe der äußeren Schleife, $\leq n$ mal innere Schleife
            \item Komplexitätsklasse: \osquare
        \end{itemize}
    \end{block}
\end{frame}

\begin{frame}
    \begin{block}{Beobachtungen}
        \begin{itemize}
            \item Komplexitätsklassen geben nur die Größenordnung an.
            \item Konstante Faktoren und nicht-dominante Terme werden vernachlässigt.
        \end{itemize}
    \end{block}
    \begin{block}<2->{Beispiele}
        \vspace{-3ex}
        \begin{align*}
            \olin &= \bigo{2n} = \bigo{\frac{n}{2}} \\
            \osquare &= \bigo{n^2+n+1} = \bigo{\left(\frac{n}{2}\right)^2} \\
            \onlog &= \bigo{2n \log n  + 50n}
        \end{align*}
    \end{block}
\end{frame}

\begin{frame}
    \begin{block}{Intuition:}
        \begin{itemize}
            \item Der Unterschied zwischen \olin und \bigo{2n} kann durch schnellere Hardware ausgeglichen werden.
            \item Ebenso der Unterschied zwischen \osquare und \bigo{2(n^2)}.
            \item Der Unterschied zwischen \olin und \osquare kann nicht so einfach kompensiert werden.
            \item Das Verhalten von Polynomen (Funktionen) wird i.W. vom \emph{Leitterm} bestimmt.
        \end{itemize}
    \end{block}
    \begin{block}<2->{Ziel bei der Entwicklung:}
        \begin{itemize}
            \item Komplexitätsklasse möglichst klein halten.
            \item \alert{Komplexität kann nicht durch Hardware ausgeglichen werden!}
            \item Konstante oder lineare Faktoren sind weniger von Bedeutung.
        \end{itemize}
    \end{block}
\end{frame}
    
\begin{frame}
    \begin{defblock}{$O$-Komplexität}
        Gegeben eine Funktion $f(n)$, ist $f(n) = O(g(n))$ genau dann, wenn es eine positive Konstante $c$ gibt, so dass für alle $n \geq n_0$ gilt:
        \begin{align*}
            f(n) \leq c \cdot g(n)
        \end{align*}
    \end{defblock}
    \begin{block}<2->{Intuitiv:}
        \begin{itemize}
            \item Falls $f(n) \geq g(n)$ für alle $n$ gilt, dann unterscheiden sich
                  die Funktionen nur durch einen konstanten Faktor.
            \item Für große $n$ ist $g(n)$ eine gute Abschätzung für $f(n)$.
        \end{itemize}
    \end{block}
\end{frame}

\begin{frame}
    \begin{defblock}{$O$-Komplexität}
        \vspace{-3ex}
        \begin{align*}
            f \in \bigo{g} &\Leftrightarrow \exists_{c>0} \exists_{n_0} \forall_{n \geq n_0}: f(n) \leq c \cdot g(n)
        \end{align*}
    \end{defblock}
    \begin{block}{Intuitiv:}
        \begin{itemize}
            \item Die Funktion $f(n)$ wächst nicht schneller als $g(n)$.
            \item Für \alert{fast alle $n$} gilt $f(n) \leq c \cdot g(n)$.
        \end{itemize}
    \end{block}
    \begin{block}<2->{Konstante Faktoren sind nicht relevant:}
        \begin{itemize}
            \item Bewegen sich im Bereich der Ungenauigkeit, die durch unterschiedliche Hardware entsteht.
            \item Bieten geringes Optimierungspotenzial.
            \item Können ggf. durch Hardware ausgeglichen werden.
        \end{itemize}
    \end{block}
\end{frame}
